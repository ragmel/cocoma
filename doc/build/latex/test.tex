% Generated by Sphinx.
\def\sphinxdocclass{report}
\documentclass[letterpaper,10pt,english]{sphinxmanual}
\usepackage[utf8]{inputenc}
\DeclareUnicodeCharacter{00A0}{\nobreakspace}
\usepackage[T1]{fontenc}
\usepackage{babel}
\usepackage{times}
\usepackage[Bjarne]{fncychap}
\usepackage{longtable}
\usepackage{sphinx}
\usepackage{multirow}


\title{COCOMA Documentation}
\date{February 14, 2013}
\release{1}
\author{Sergej Svorobej}
\newcommand{\sphinxlogo}{}
\renewcommand{\releasename}{Release}
\makeindex

\makeatletter
\def\PYG@reset{\let\PYG@it=\relax \let\PYG@bf=\relax%
    \let\PYG@ul=\relax \let\PYG@tc=\relax%
    \let\PYG@bc=\relax \let\PYG@ff=\relax}
\def\PYG@tok#1{\csname PYG@tok@#1\endcsname}
\def\PYG@toks#1+{\ifx\relax#1\empty\else%
    \PYG@tok{#1}\expandafter\PYG@toks\fi}
\def\PYG@do#1{\PYG@bc{\PYG@tc{\PYG@ul{%
    \PYG@it{\PYG@bf{\PYG@ff{#1}}}}}}}
\def\PYG#1#2{\PYG@reset\PYG@toks#1+\relax+\PYG@do{#2}}

\expandafter\def\csname PYG@tok@gd\endcsname{\def\PYG@tc##1{\textcolor[rgb]{0.63,0.00,0.00}{##1}}}
\expandafter\def\csname PYG@tok@gu\endcsname{\let\PYG@bf=\textbf\def\PYG@tc##1{\textcolor[rgb]{0.50,0.00,0.50}{##1}}}
\expandafter\def\csname PYG@tok@gt\endcsname{\def\PYG@tc##1{\textcolor[rgb]{0.00,0.27,0.87}{##1}}}
\expandafter\def\csname PYG@tok@gs\endcsname{\let\PYG@bf=\textbf}
\expandafter\def\csname PYG@tok@gr\endcsname{\def\PYG@tc##1{\textcolor[rgb]{1.00,0.00,0.00}{##1}}}
\expandafter\def\csname PYG@tok@cm\endcsname{\let\PYG@it=\textit\def\PYG@tc##1{\textcolor[rgb]{0.25,0.50,0.56}{##1}}}
\expandafter\def\csname PYG@tok@vg\endcsname{\def\PYG@tc##1{\textcolor[rgb]{0.73,0.38,0.84}{##1}}}
\expandafter\def\csname PYG@tok@m\endcsname{\def\PYG@tc##1{\textcolor[rgb]{0.13,0.50,0.31}{##1}}}
\expandafter\def\csname PYG@tok@mh\endcsname{\def\PYG@tc##1{\textcolor[rgb]{0.13,0.50,0.31}{##1}}}
\expandafter\def\csname PYG@tok@cs\endcsname{\def\PYG@tc##1{\textcolor[rgb]{0.25,0.50,0.56}{##1}}\def\PYG@bc##1{\setlength{\fboxsep}{0pt}\colorbox[rgb]{1.00,0.94,0.94}{\strut ##1}}}
\expandafter\def\csname PYG@tok@ge\endcsname{\let\PYG@it=\textit}
\expandafter\def\csname PYG@tok@vc\endcsname{\def\PYG@tc##1{\textcolor[rgb]{0.73,0.38,0.84}{##1}}}
\expandafter\def\csname PYG@tok@il\endcsname{\def\PYG@tc##1{\textcolor[rgb]{0.13,0.50,0.31}{##1}}}
\expandafter\def\csname PYG@tok@go\endcsname{\def\PYG@tc##1{\textcolor[rgb]{0.20,0.20,0.20}{##1}}}
\expandafter\def\csname PYG@tok@cp\endcsname{\def\PYG@tc##1{\textcolor[rgb]{0.00,0.44,0.13}{##1}}}
\expandafter\def\csname PYG@tok@gi\endcsname{\def\PYG@tc##1{\textcolor[rgb]{0.00,0.63,0.00}{##1}}}
\expandafter\def\csname PYG@tok@gh\endcsname{\let\PYG@bf=\textbf\def\PYG@tc##1{\textcolor[rgb]{0.00,0.00,0.50}{##1}}}
\expandafter\def\csname PYG@tok@ni\endcsname{\let\PYG@bf=\textbf\def\PYG@tc##1{\textcolor[rgb]{0.84,0.33,0.22}{##1}}}
\expandafter\def\csname PYG@tok@nl\endcsname{\let\PYG@bf=\textbf\def\PYG@tc##1{\textcolor[rgb]{0.00,0.13,0.44}{##1}}}
\expandafter\def\csname PYG@tok@nn\endcsname{\let\PYG@bf=\textbf\def\PYG@tc##1{\textcolor[rgb]{0.05,0.52,0.71}{##1}}}
\expandafter\def\csname PYG@tok@no\endcsname{\def\PYG@tc##1{\textcolor[rgb]{0.38,0.68,0.84}{##1}}}
\expandafter\def\csname PYG@tok@na\endcsname{\def\PYG@tc##1{\textcolor[rgb]{0.25,0.44,0.63}{##1}}}
\expandafter\def\csname PYG@tok@nb\endcsname{\def\PYG@tc##1{\textcolor[rgb]{0.00,0.44,0.13}{##1}}}
\expandafter\def\csname PYG@tok@nc\endcsname{\let\PYG@bf=\textbf\def\PYG@tc##1{\textcolor[rgb]{0.05,0.52,0.71}{##1}}}
\expandafter\def\csname PYG@tok@nd\endcsname{\let\PYG@bf=\textbf\def\PYG@tc##1{\textcolor[rgb]{0.33,0.33,0.33}{##1}}}
\expandafter\def\csname PYG@tok@ne\endcsname{\def\PYG@tc##1{\textcolor[rgb]{0.00,0.44,0.13}{##1}}}
\expandafter\def\csname PYG@tok@nf\endcsname{\def\PYG@tc##1{\textcolor[rgb]{0.02,0.16,0.49}{##1}}}
\expandafter\def\csname PYG@tok@si\endcsname{\let\PYG@it=\textit\def\PYG@tc##1{\textcolor[rgb]{0.44,0.63,0.82}{##1}}}
\expandafter\def\csname PYG@tok@s2\endcsname{\def\PYG@tc##1{\textcolor[rgb]{0.25,0.44,0.63}{##1}}}
\expandafter\def\csname PYG@tok@vi\endcsname{\def\PYG@tc##1{\textcolor[rgb]{0.73,0.38,0.84}{##1}}}
\expandafter\def\csname PYG@tok@nt\endcsname{\let\PYG@bf=\textbf\def\PYG@tc##1{\textcolor[rgb]{0.02,0.16,0.45}{##1}}}
\expandafter\def\csname PYG@tok@nv\endcsname{\def\PYG@tc##1{\textcolor[rgb]{0.73,0.38,0.84}{##1}}}
\expandafter\def\csname PYG@tok@s1\endcsname{\def\PYG@tc##1{\textcolor[rgb]{0.25,0.44,0.63}{##1}}}
\expandafter\def\csname PYG@tok@gp\endcsname{\let\PYG@bf=\textbf\def\PYG@tc##1{\textcolor[rgb]{0.78,0.36,0.04}{##1}}}
\expandafter\def\csname PYG@tok@sh\endcsname{\def\PYG@tc##1{\textcolor[rgb]{0.25,0.44,0.63}{##1}}}
\expandafter\def\csname PYG@tok@ow\endcsname{\let\PYG@bf=\textbf\def\PYG@tc##1{\textcolor[rgb]{0.00,0.44,0.13}{##1}}}
\expandafter\def\csname PYG@tok@sx\endcsname{\def\PYG@tc##1{\textcolor[rgb]{0.78,0.36,0.04}{##1}}}
\expandafter\def\csname PYG@tok@bp\endcsname{\def\PYG@tc##1{\textcolor[rgb]{0.00,0.44,0.13}{##1}}}
\expandafter\def\csname PYG@tok@c1\endcsname{\let\PYG@it=\textit\def\PYG@tc##1{\textcolor[rgb]{0.25,0.50,0.56}{##1}}}
\expandafter\def\csname PYG@tok@kc\endcsname{\let\PYG@bf=\textbf\def\PYG@tc##1{\textcolor[rgb]{0.00,0.44,0.13}{##1}}}
\expandafter\def\csname PYG@tok@c\endcsname{\let\PYG@it=\textit\def\PYG@tc##1{\textcolor[rgb]{0.25,0.50,0.56}{##1}}}
\expandafter\def\csname PYG@tok@mf\endcsname{\def\PYG@tc##1{\textcolor[rgb]{0.13,0.50,0.31}{##1}}}
\expandafter\def\csname PYG@tok@err\endcsname{\def\PYG@bc##1{\setlength{\fboxsep}{0pt}\fcolorbox[rgb]{1.00,0.00,0.00}{1,1,1}{\strut ##1}}}
\expandafter\def\csname PYG@tok@kd\endcsname{\let\PYG@bf=\textbf\def\PYG@tc##1{\textcolor[rgb]{0.00,0.44,0.13}{##1}}}
\expandafter\def\csname PYG@tok@ss\endcsname{\def\PYG@tc##1{\textcolor[rgb]{0.32,0.47,0.09}{##1}}}
\expandafter\def\csname PYG@tok@sr\endcsname{\def\PYG@tc##1{\textcolor[rgb]{0.14,0.33,0.53}{##1}}}
\expandafter\def\csname PYG@tok@mo\endcsname{\def\PYG@tc##1{\textcolor[rgb]{0.13,0.50,0.31}{##1}}}
\expandafter\def\csname PYG@tok@mi\endcsname{\def\PYG@tc##1{\textcolor[rgb]{0.13,0.50,0.31}{##1}}}
\expandafter\def\csname PYG@tok@kn\endcsname{\let\PYG@bf=\textbf\def\PYG@tc##1{\textcolor[rgb]{0.00,0.44,0.13}{##1}}}
\expandafter\def\csname PYG@tok@o\endcsname{\def\PYG@tc##1{\textcolor[rgb]{0.40,0.40,0.40}{##1}}}
\expandafter\def\csname PYG@tok@kr\endcsname{\let\PYG@bf=\textbf\def\PYG@tc##1{\textcolor[rgb]{0.00,0.44,0.13}{##1}}}
\expandafter\def\csname PYG@tok@s\endcsname{\def\PYG@tc##1{\textcolor[rgb]{0.25,0.44,0.63}{##1}}}
\expandafter\def\csname PYG@tok@kp\endcsname{\def\PYG@tc##1{\textcolor[rgb]{0.00,0.44,0.13}{##1}}}
\expandafter\def\csname PYG@tok@w\endcsname{\def\PYG@tc##1{\textcolor[rgb]{0.73,0.73,0.73}{##1}}}
\expandafter\def\csname PYG@tok@kt\endcsname{\def\PYG@tc##1{\textcolor[rgb]{0.56,0.13,0.00}{##1}}}
\expandafter\def\csname PYG@tok@sc\endcsname{\def\PYG@tc##1{\textcolor[rgb]{0.25,0.44,0.63}{##1}}}
\expandafter\def\csname PYG@tok@sb\endcsname{\def\PYG@tc##1{\textcolor[rgb]{0.25,0.44,0.63}{##1}}}
\expandafter\def\csname PYG@tok@k\endcsname{\let\PYG@bf=\textbf\def\PYG@tc##1{\textcolor[rgb]{0.00,0.44,0.13}{##1}}}
\expandafter\def\csname PYG@tok@se\endcsname{\let\PYG@bf=\textbf\def\PYG@tc##1{\textcolor[rgb]{0.25,0.44,0.63}{##1}}}
\expandafter\def\csname PYG@tok@sd\endcsname{\let\PYG@it=\textit\def\PYG@tc##1{\textcolor[rgb]{0.25,0.44,0.63}{##1}}}

\def\PYGZbs{\char`\\}
\def\PYGZus{\char`\_}
\def\PYGZob{\char`\{}
\def\PYGZcb{\char`\}}
\def\PYGZca{\char`\^}
\def\PYGZam{\char`\&}
\def\PYGZlt{\char`\<}
\def\PYGZgt{\char`\>}
\def\PYGZsh{\char`\#}
\def\PYGZpc{\char`\%}
\def\PYGZdl{\char`\$}
\def\PYGZhy{\char`\-}
\def\PYGZsq{\char`\'}
\def\PYGZdq{\char`\"}
\def\PYGZti{\char`\~}
% for compatibility with earlier versions
\def\PYGZat{@}
\def\PYGZlb{[}
\def\PYGZrb{]}
\makeatother

\begin{document}

\maketitle
\tableofcontents
\phantomsection\label{index::doc}


COCOMA framework was designed by SAP as part of EU funded BonFIRE project. Task of COCOMA framework is
to create, monitor and control contentious and malicious system workload. By using
this framework experimenters are able to make testing process more accurate
and anticipate various scenarios of cloud infrastructure behaviour, collect and
correlate metrics of the emulated environment with the test results.


\chapter{Contents}
\label{index:controlled-contentious-and-malicious-cocoma-framework-1-0}\label{index:contents}

\section{How to use it}
\label{01_how_to_use_it:how-to-use-it}\label{01_how_to_use_it::doc}
That has a paragraph about a main subject and is set when the `='
is at least the same length of the title itself.


\subsection{Subject Subtitle}
\label{01_how_to_use_it:subject-subtitle}
Subtitles are set with `-` and are required to have the same length
of the subttitle itself, just like titles.

Lists can be unnumbered like:
\begin{itemize}
\item {} 
Item Foo

\item {} 
Item Bar

\end{itemize}

Or automatically numbered:
\begin{enumerate}
\item {} 
Item 1

\item {} 
Item 2

\end{enumerate}


\subsection{Inline Markup}
\label{01_how_to_use_it:inline-markup}
Words can have \emph{emphasis in italics} or be \textbf{bold} and you can
define code samples with back quotes, like when you talk about a
command: \code{sudo} gives you super user powers!

This is an example on how to link images:

\begin{Verbatim}[commandchars=\\\{\},numbers=left,firstnumber=1,stepnumber=1]
         \PYG{n+nt}{\PYGZlt{}emulation}\PYG{n+nt}{\PYGZgt{}}
           \PYG{n+nt}{\PYGZlt{}emuname}\PYG{n+nt}{\PYGZgt{}}CPU\PYGZus{}emu\PYG{n+nt}{\PYGZlt{}/emuname\PYGZgt{}}
           \PYG{n+nt}{\PYGZlt{}emuType}\PYG{n+nt}{\PYGZgt{}}Mix\PYG{n+nt}{\PYGZlt{}/emuType\PYGZgt{}}
           \PYG{n+nt}{\PYGZlt{}emuresourceType}\PYG{n+nt}{\PYGZgt{}}CPU\PYG{n+nt}{\PYGZlt{}/emuresourceType\PYGZgt{}}
           \PYG{n+nt}{\PYGZlt{}emustartTime}\PYG{n+nt}{\PYGZgt{}}now\PYG{n+nt}{\PYGZlt{}/emustartTime\PYGZgt{}}
           \PYG{c}{\PYGZlt{}!\PYGZhy{}\PYGZhy{}}\PYG{c}{duration in seconds }\PYG{c}{\PYGZhy{}\PYGZhy{}\PYGZgt{}}
           \PYG{n+nt}{\PYGZlt{}emustopTime}\PYG{n+nt}{\PYGZgt{}}15\PYG{n+nt}{\PYGZlt{}/emustopTime\PYGZgt{}}

           \PYG{n+nt}{\PYGZlt{}distributions}\PYG{n+nt}{\PYGZgt{}}
            \PYG{n+nt}{\PYGZlt{}name}\PYG{n+nt}{\PYGZgt{}}CPU\PYGZus{}distro\PYG{n+nt}{\PYGZlt{}/name\PYGZgt{}}
            \PYG{n+nt}{\PYGZlt{}startTime}\PYG{n+nt}{\PYGZgt{}}0\PYG{n+nt}{\PYGZlt{}/startTime\PYGZgt{}}
            \PYG{c}{\PYGZlt{}!\PYGZhy{}\PYGZhy{}}\PYG{c}{duration in seconds }\PYG{c}{\PYGZhy{}\PYGZhy{}\PYGZgt{}}
            \PYG{n+nt}{\PYGZlt{}duration}\PYG{n+nt}{\PYGZgt{}}10\PYG{n+nt}{\PYGZlt{}/duration\PYGZgt{}}
            \PYG{n+nt}{\PYGZlt{}granularity}\PYG{n+nt}{\PYGZgt{}}1\PYG{n+nt}{\PYGZlt{}/granularity\PYGZgt{}}
            \PYG{n+nt}{\PYGZlt{}distribution} \PYG{n+na}{href=}\PYG{l+s}{\PYGZdq{}/distributions/linear\PYGZdq{}} \PYG{n+na}{name=}\PYG{l+s}{\PYGZdq{}linear\PYGZdq{}} \PYG{n+nt}{/\PYGZgt{}}
            \PYG{c}{\PYGZlt{}!\PYGZhy{}\PYGZhy{}}\PYG{c}{cpu utilization distribution range}\PYG{c}{\PYGZhy{}\PYGZhy{}\PYGZgt{}}
            \PYG{n+nt}{\PYGZlt{}startLoad}\PYG{n+nt}{\PYGZgt{}}10\PYG{n+nt}{\PYGZlt{}/startLoad\PYGZgt{}}
            \PYG{n+nt}{\PYGZlt{}stopLoad}\PYG{n+nt}{\PYGZgt{}}95\PYG{n+nt}{\PYGZlt{}/stopLoad\PYGZgt{}}
            \PYG{n+nt}{\PYGZlt{}emulator} \PYG{n+na}{href=}\PYG{l+s}{\PYGZdq{}/emulators/lookbusy\PYGZdq{}} \PYG{n+na}{name=}\PYG{l+s}{\PYGZdq{}lookbusy\PYGZdq{}} \PYG{n+nt}{/\PYGZgt{}}
            \PYG{n+nt}{\PYGZlt{}emulator\PYGZhy{}params}\PYG{n+nt}{\PYGZgt{}}
                 \PYG{c}{\PYGZlt{}!\PYGZhy{}\PYGZhy{}}\PYG{c}{more parameters will be added }\PYG{c}{\PYGZhy{}\PYGZhy{}\PYGZgt{}}
                 \PYG{n+nt}{\PYGZlt{}resourceType}\PYG{n+nt}{\PYGZgt{}}CPU\PYG{n+nt}{\PYGZlt{}/resourceType\PYGZgt{}}
                 \PYG{c}{\PYGZlt{}!\PYGZhy{}\PYGZhy{}}\PYG{c}{Number of CPUs to keep busy (default: autodetected)}\PYG{c}{\PYGZhy{}\PYGZhy{}\PYGZgt{}}
                 \PYG{n+nt}{\PYGZlt{}ncpus}\PYG{n+nt}{\PYGZgt{}}0\PYG{n+nt}{\PYGZlt{}/ncpus\PYGZgt{}}
            \PYG{n+nt}{\PYGZlt{}/emulator\PYGZhy{}params\PYGZgt{}}
          \PYG{n+nt}{\PYGZlt{}/distributions\PYGZgt{}}

          \PYG{n+nt}{\PYGZlt{}log}\PYG{n+nt}{\PYGZgt{}}
            \PYG{c}{\PYGZlt{}!\PYGZhy{}\PYGZhy{}}\PYG{c}{ Use value \PYGZdq{}1\PYGZdq{} to enable logging(by default logging is off)  }\PYG{c}{\PYGZhy{}\PYGZhy{}\PYGZgt{}}
            \PYG{n+nt}{\PYGZlt{}enable}\PYG{n+nt}{\PYGZgt{}}1\PYG{n+nt}{\PYGZlt{}/enable\PYGZgt{}}
            \PYG{c}{\PYGZlt{}!\PYGZhy{}\PYGZhy{}}\PYG{c}{ Use seconds for setting probe intervals(if logging is enabled default is 3sec)  }\PYG{c}{\PYGZhy{}\PYGZhy{}\PYGZgt{}}
            \PYG{n+nt}{\PYGZlt{}frequency}\PYG{n+nt}{\PYGZgt{}}3\PYG{n+nt}{\PYGZlt{}/frequency\PYGZgt{}}
          \PYG{n+nt}{\PYGZlt{}/log\PYGZgt{}}
         \PYG{n+nt}{\PYGZlt{}/emulation\PYGZgt{}}
\end{Verbatim}


\chapter{Indices and tables}
\label{index:indices-and-tables}\label{index::doc}\begin{itemize}
\item {} 
\emph{genindex}

\item {} 
\emph{search}

\end{itemize}



\renewcommand{\indexname}{Index}
\printindex
\end{document}
