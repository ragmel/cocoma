% Generated by Sphinx.
\def\sphinxdocclass{report}
\documentclass[letterpaper,10pt,english]{sphinxmanual}
\usepackage[utf8]{inputenc}
\DeclareUnicodeCharacter{00A0}{\nobreakspace}
\usepackage[T1]{fontenc}
\usepackage{babel}
\usepackage{times}
\usepackage[Bjarne]{fncychap}
\usepackage{longtable}
\usepackage{sphinx}
\usepackage{multirow}


\title{COCOMA Documentation}
\date{April 05, 2013}
\release{1}
\author{Sergej Svorobej}
\newcommand{\sphinxlogo}{}
\renewcommand{\releasename}{Release}
\makeindex

\makeatletter
\def\PYG@reset{\let\PYG@it=\relax \let\PYG@bf=\relax%
    \let\PYG@ul=\relax \let\PYG@tc=\relax%
    \let\PYG@bc=\relax \let\PYG@ff=\relax}
\def\PYG@tok#1{\csname PYG@tok@#1\endcsname}
\def\PYG@toks#1+{\ifx\relax#1\empty\else%
    \PYG@tok{#1}\expandafter\PYG@toks\fi}
\def\PYG@do#1{\PYG@bc{\PYG@tc{\PYG@ul{%
    \PYG@it{\PYG@bf{\PYG@ff{#1}}}}}}}
\def\PYG#1#2{\PYG@reset\PYG@toks#1+\relax+\PYG@do{#2}}

\expandafter\def\csname PYG@tok@gd\endcsname{\def\PYG@tc##1{\textcolor[rgb]{0.63,0.00,0.00}{##1}}}
\expandafter\def\csname PYG@tok@gu\endcsname{\let\PYG@bf=\textbf\def\PYG@tc##1{\textcolor[rgb]{0.50,0.00,0.50}{##1}}}
\expandafter\def\csname PYG@tok@gt\endcsname{\def\PYG@tc##1{\textcolor[rgb]{0.00,0.27,0.87}{##1}}}
\expandafter\def\csname PYG@tok@gs\endcsname{\let\PYG@bf=\textbf}
\expandafter\def\csname PYG@tok@gr\endcsname{\def\PYG@tc##1{\textcolor[rgb]{1.00,0.00,0.00}{##1}}}
\expandafter\def\csname PYG@tok@cm\endcsname{\let\PYG@it=\textit\def\PYG@tc##1{\textcolor[rgb]{0.25,0.50,0.56}{##1}}}
\expandafter\def\csname PYG@tok@vg\endcsname{\def\PYG@tc##1{\textcolor[rgb]{0.73,0.38,0.84}{##1}}}
\expandafter\def\csname PYG@tok@m\endcsname{\def\PYG@tc##1{\textcolor[rgb]{0.13,0.50,0.31}{##1}}}
\expandafter\def\csname PYG@tok@mh\endcsname{\def\PYG@tc##1{\textcolor[rgb]{0.13,0.50,0.31}{##1}}}
\expandafter\def\csname PYG@tok@cs\endcsname{\def\PYG@tc##1{\textcolor[rgb]{0.25,0.50,0.56}{##1}}\def\PYG@bc##1{\setlength{\fboxsep}{0pt}\colorbox[rgb]{1.00,0.94,0.94}{\strut ##1}}}
\expandafter\def\csname PYG@tok@ge\endcsname{\let\PYG@it=\textit}
\expandafter\def\csname PYG@tok@vc\endcsname{\def\PYG@tc##1{\textcolor[rgb]{0.73,0.38,0.84}{##1}}}
\expandafter\def\csname PYG@tok@il\endcsname{\def\PYG@tc##1{\textcolor[rgb]{0.13,0.50,0.31}{##1}}}
\expandafter\def\csname PYG@tok@go\endcsname{\def\PYG@tc##1{\textcolor[rgb]{0.20,0.20,0.20}{##1}}}
\expandafter\def\csname PYG@tok@cp\endcsname{\def\PYG@tc##1{\textcolor[rgb]{0.00,0.44,0.13}{##1}}}
\expandafter\def\csname PYG@tok@gi\endcsname{\def\PYG@tc##1{\textcolor[rgb]{0.00,0.63,0.00}{##1}}}
\expandafter\def\csname PYG@tok@gh\endcsname{\let\PYG@bf=\textbf\def\PYG@tc##1{\textcolor[rgb]{0.00,0.00,0.50}{##1}}}
\expandafter\def\csname PYG@tok@ni\endcsname{\let\PYG@bf=\textbf\def\PYG@tc##1{\textcolor[rgb]{0.84,0.33,0.22}{##1}}}
\expandafter\def\csname PYG@tok@nl\endcsname{\let\PYG@bf=\textbf\def\PYG@tc##1{\textcolor[rgb]{0.00,0.13,0.44}{##1}}}
\expandafter\def\csname PYG@tok@nn\endcsname{\let\PYG@bf=\textbf\def\PYG@tc##1{\textcolor[rgb]{0.05,0.52,0.71}{##1}}}
\expandafter\def\csname PYG@tok@no\endcsname{\def\PYG@tc##1{\textcolor[rgb]{0.38,0.68,0.84}{##1}}}
\expandafter\def\csname PYG@tok@na\endcsname{\def\PYG@tc##1{\textcolor[rgb]{0.25,0.44,0.63}{##1}}}
\expandafter\def\csname PYG@tok@nb\endcsname{\def\PYG@tc##1{\textcolor[rgb]{0.00,0.44,0.13}{##1}}}
\expandafter\def\csname PYG@tok@nc\endcsname{\let\PYG@bf=\textbf\def\PYG@tc##1{\textcolor[rgb]{0.05,0.52,0.71}{##1}}}
\expandafter\def\csname PYG@tok@nd\endcsname{\let\PYG@bf=\textbf\def\PYG@tc##1{\textcolor[rgb]{0.33,0.33,0.33}{##1}}}
\expandafter\def\csname PYG@tok@ne\endcsname{\def\PYG@tc##1{\textcolor[rgb]{0.00,0.44,0.13}{##1}}}
\expandafter\def\csname PYG@tok@nf\endcsname{\def\PYG@tc##1{\textcolor[rgb]{0.02,0.16,0.49}{##1}}}
\expandafter\def\csname PYG@tok@si\endcsname{\let\PYG@it=\textit\def\PYG@tc##1{\textcolor[rgb]{0.44,0.63,0.82}{##1}}}
\expandafter\def\csname PYG@tok@s2\endcsname{\def\PYG@tc##1{\textcolor[rgb]{0.25,0.44,0.63}{##1}}}
\expandafter\def\csname PYG@tok@vi\endcsname{\def\PYG@tc##1{\textcolor[rgb]{0.73,0.38,0.84}{##1}}}
\expandafter\def\csname PYG@tok@nt\endcsname{\let\PYG@bf=\textbf\def\PYG@tc##1{\textcolor[rgb]{0.02,0.16,0.45}{##1}}}
\expandafter\def\csname PYG@tok@nv\endcsname{\def\PYG@tc##1{\textcolor[rgb]{0.73,0.38,0.84}{##1}}}
\expandafter\def\csname PYG@tok@s1\endcsname{\def\PYG@tc##1{\textcolor[rgb]{0.25,0.44,0.63}{##1}}}
\expandafter\def\csname PYG@tok@gp\endcsname{\let\PYG@bf=\textbf\def\PYG@tc##1{\textcolor[rgb]{0.78,0.36,0.04}{##1}}}
\expandafter\def\csname PYG@tok@sh\endcsname{\def\PYG@tc##1{\textcolor[rgb]{0.25,0.44,0.63}{##1}}}
\expandafter\def\csname PYG@tok@ow\endcsname{\let\PYG@bf=\textbf\def\PYG@tc##1{\textcolor[rgb]{0.00,0.44,0.13}{##1}}}
\expandafter\def\csname PYG@tok@sx\endcsname{\def\PYG@tc##1{\textcolor[rgb]{0.78,0.36,0.04}{##1}}}
\expandafter\def\csname PYG@tok@bp\endcsname{\def\PYG@tc##1{\textcolor[rgb]{0.00,0.44,0.13}{##1}}}
\expandafter\def\csname PYG@tok@c1\endcsname{\let\PYG@it=\textit\def\PYG@tc##1{\textcolor[rgb]{0.25,0.50,0.56}{##1}}}
\expandafter\def\csname PYG@tok@kc\endcsname{\let\PYG@bf=\textbf\def\PYG@tc##1{\textcolor[rgb]{0.00,0.44,0.13}{##1}}}
\expandafter\def\csname PYG@tok@c\endcsname{\let\PYG@it=\textit\def\PYG@tc##1{\textcolor[rgb]{0.25,0.50,0.56}{##1}}}
\expandafter\def\csname PYG@tok@mf\endcsname{\def\PYG@tc##1{\textcolor[rgb]{0.13,0.50,0.31}{##1}}}
\expandafter\def\csname PYG@tok@err\endcsname{\def\PYG@bc##1{\setlength{\fboxsep}{0pt}\fcolorbox[rgb]{1.00,0.00,0.00}{1,1,1}{\strut ##1}}}
\expandafter\def\csname PYG@tok@kd\endcsname{\let\PYG@bf=\textbf\def\PYG@tc##1{\textcolor[rgb]{0.00,0.44,0.13}{##1}}}
\expandafter\def\csname PYG@tok@ss\endcsname{\def\PYG@tc##1{\textcolor[rgb]{0.32,0.47,0.09}{##1}}}
\expandafter\def\csname PYG@tok@sr\endcsname{\def\PYG@tc##1{\textcolor[rgb]{0.14,0.33,0.53}{##1}}}
\expandafter\def\csname PYG@tok@mo\endcsname{\def\PYG@tc##1{\textcolor[rgb]{0.13,0.50,0.31}{##1}}}
\expandafter\def\csname PYG@tok@mi\endcsname{\def\PYG@tc##1{\textcolor[rgb]{0.13,0.50,0.31}{##1}}}
\expandafter\def\csname PYG@tok@kn\endcsname{\let\PYG@bf=\textbf\def\PYG@tc##1{\textcolor[rgb]{0.00,0.44,0.13}{##1}}}
\expandafter\def\csname PYG@tok@o\endcsname{\def\PYG@tc##1{\textcolor[rgb]{0.40,0.40,0.40}{##1}}}
\expandafter\def\csname PYG@tok@kr\endcsname{\let\PYG@bf=\textbf\def\PYG@tc##1{\textcolor[rgb]{0.00,0.44,0.13}{##1}}}
\expandafter\def\csname PYG@tok@s\endcsname{\def\PYG@tc##1{\textcolor[rgb]{0.25,0.44,0.63}{##1}}}
\expandafter\def\csname PYG@tok@kp\endcsname{\def\PYG@tc##1{\textcolor[rgb]{0.00,0.44,0.13}{##1}}}
\expandafter\def\csname PYG@tok@w\endcsname{\def\PYG@tc##1{\textcolor[rgb]{0.73,0.73,0.73}{##1}}}
\expandafter\def\csname PYG@tok@kt\endcsname{\def\PYG@tc##1{\textcolor[rgb]{0.56,0.13,0.00}{##1}}}
\expandafter\def\csname PYG@tok@sc\endcsname{\def\PYG@tc##1{\textcolor[rgb]{0.25,0.44,0.63}{##1}}}
\expandafter\def\csname PYG@tok@sb\endcsname{\def\PYG@tc##1{\textcolor[rgb]{0.25,0.44,0.63}{##1}}}
\expandafter\def\csname PYG@tok@k\endcsname{\let\PYG@bf=\textbf\def\PYG@tc##1{\textcolor[rgb]{0.00,0.44,0.13}{##1}}}
\expandafter\def\csname PYG@tok@se\endcsname{\let\PYG@bf=\textbf\def\PYG@tc##1{\textcolor[rgb]{0.25,0.44,0.63}{##1}}}
\expandafter\def\csname PYG@tok@sd\endcsname{\let\PYG@it=\textit\def\PYG@tc##1{\textcolor[rgb]{0.25,0.44,0.63}{##1}}}

\def\PYGZbs{\char`\\}
\def\PYGZus{\char`\_}
\def\PYGZob{\char`\{}
\def\PYGZcb{\char`\}}
\def\PYGZca{\char`\^}
\def\PYGZam{\char`\&}
\def\PYGZlt{\char`\<}
\def\PYGZgt{\char`\>}
\def\PYGZsh{\char`\#}
\def\PYGZpc{\char`\%}
\def\PYGZdl{\char`\$}
\def\PYGZhy{\char`\-}
\def\PYGZsq{\char`\'}
\def\PYGZdq{\char`\"}
\def\PYGZti{\char`\~}
% for compatibility with earlier versions
\def\PYGZat{@}
\def\PYGZlb{[}
\def\PYGZrb{]}
\makeatother

\begin{document}

\maketitle
\tableofcontents
\phantomsection\label{index::doc}


COCOMA framework was designed by SAP as part of EU funded BonFIRE project. Task of COCOMA framework is
to create, monitor and control contentious and malicious system workload. By using
this framework experimenters are able to make testing process more accurate
and anticipate various scenarios of cloud infrastructure behaviour, collect and
correlate metrics of the emulated environment with the test results.


\chapter{Contents}
\label{index:controlled-contentious-and-malicious-cocoma-framework-1-0}\label{index:contents}

\section{How to use it}
\label{01_how_to_use_it:how-to-use-it}\label{01_how_to_use_it::doc}
In order to use COCOMA framework experimenter creates an emulation using XML language(see below Examples section). Emulation should contain all the neccessary information
about duration, magnitude and required resource usage. Once XML document is received by COCOMA, the framework will automatically schedule and execute
required workload on the chosen resource(-s) such as CPU, IO, Memory and/or Network.


\subsection{Installation}
\label{01_how_to_use_it:installation}
The framework is designed to run on GNU/Linux and released in \emph{.deb} package only.
Once you have downloaded latest COCOMA version install it by running:

\begin{Verbatim}[commandchars=\\\{\}]
\PYGZdl{} dpkg \PYGZhy{}i cocoma\PYGZus{}X.X\PYGZhy{}X\PYGZus{}all.deb
\end{Verbatim}

The application will be installed to folder \emph{``/usr/share/pyshared/cocoma''}. All the additional required programs and libraries will be downloaded and installed on the fly if missing.
To check check if it was installed correctly run:

\begin{Verbatim}[commandchars=\\\{\}]
\PYGZdl{} ccmsh \PYGZhy{}v
\end{Verbatim}


\subsection{Starting Components}
\label{01_how_to_use_it:starting-components}
To avail full functionality of COCOMA two daemons need to be started:
\begin{itemize}
\item {} 
Scheduler daemon (mandatory)

\item {} 
API Daemon (optional if REST API functionality is required)

\end{itemize}

\textbf{Scheduler daemon} - runs in the background and executes workload with differential parameters at the time defined in the emulation properties.
to start scheduler use command:

\begin{Verbatim}[commandchars=\\\{\}]
\PYGZdl{} ccmsh \PYGZhy{}\PYGZhy{}start scheduler
\end{Verbatim}

Default network interface is \emph{eth0}, port \emph{51889} you can change that by adding required interface name and port number at the end:

\begin{Verbatim}[commandchars=\\\{\}]
\PYGZdl{} ccmsh \PYGZhy{}\PYGZhy{}start scheduler wlan0 5180
\end{Verbatim}

If more detailed output information is needed \emph{Scheduler} also can be started in \emph{DEBUG} mode:

\begin{Verbatim}[commandchars=\\\{\}]
\PYGZdl{} ccmsh \PYGZhy{}\PYGZhy{}start scheduler wlan0 5180 debug
\end{Verbatim}

\emph{Note: Scheduler needs to be running otherwise nothing will work. Always start it first!!}

\textbf{API daemon} - represents RESTfull web API which exposes COCOMA resources for use over the network. It follows the same startup pattern as the Scheduler:

\begin{Verbatim}[commandchars=\\\{\}]
\PYGZdl{} ccmsh \PYGZhy{}\PYGZhy{}start api
\end{Verbatim}

By default web API will try to start using \emph{eth0} network interface on port \emph{5050}, but it can be changed by supplying own parameters:

\begin{Verbatim}[commandchars=\\\{\}]
\PYGZdl{} ccmsh \PYGZhy{}\PYGZhy{}start api wlan0 3030
\end{Verbatim}

The log level will be always same as the \emph{Scheduler}.


\subsection{Command Line Arguments}
\label{01_how_to_use_it:command-line-arguments}
The COCOMA \textbf{ccmsh} command line interface has several options:
\index{ccmsh command line option!-h, --help show this help message and exit}\index{-h, --help show this help message and exit!ccmsh command line option}

\begin{fulllineitems}
\phantomsection\label{01_how_to_use_it:cmdoption-ccmsh-h}\pysigline{\bfcode{-h}\code{}\code{,~}\bfcode{--help}\code{~show~this~help~message~and~exit}}
\end{fulllineitems}

\index{ccmsh command line option!-v, --version show version information}\index{-v, --version show version information!ccmsh command line option}

\begin{fulllineitems}
\phantomsection\label{01_how_to_use_it:cmdoption-ccmsh-v}\pysigline{\bfcode{-v}\code{}\code{,~}\bfcode{--version}\code{~show~version~information}}
\end{fulllineitems}

\index{ccmsh command line option!-l, --list list all emulations or specific emulation by name}\index{-l, --list list all emulations or specific emulation by name!ccmsh command line option}

\begin{fulllineitems}
\phantomsection\label{01_how_to_use_it:cmdoption-ccmsh-l}\pysigline{\bfcode{-l}\code{}\code{,~}\bfcode{--list}\code{~list~all~emulations~or~specific~emulation~by~name}}
\end{fulllineitems}

\index{ccmsh command line option!-r, --results list all emulations results or specific emulation results by name}\index{-r, --results list all emulations results or specific emulation results by name!ccmsh command line option}

\begin{fulllineitems}
\phantomsection\label{01_how_to_use_it:cmdoption-ccmsh-r}\pysigline{\bfcode{-r}\code{}\code{,~}\bfcode{--results}\code{~list~all~emulations~results~or~specific~emulation~results~by~name}}
\end{fulllineitems}

\index{ccmsh command line option!-j, --list-jobs list of all scheduled jobs}\index{-j, --list-jobs list of all scheduled jobs!ccmsh command line option}

\begin{fulllineitems}
\phantomsection\label{01_how_to_use_it:cmdoption-ccmsh-j}\pysigline{\bfcode{-j}\code{}\code{,~}\bfcode{--list-jobs}\code{~list~of~all~scheduled~jobs}}
\end{fulllineitems}

\index{ccmsh command line option!-i, --dist lists all available distributions and gives distributiondetails by name}\index{-i, --dist lists all available distributions and gives distributiondetails by name!ccmsh command line option}

\begin{fulllineitems}
\phantomsection\label{01_how_to_use_it:cmdoption-ccmsh-i}\pysigline{\bfcode{-i}\code{}\code{,~}\bfcode{--dist}\code{~lists~all~available~distributions~and~gives~distributiondetails~by~name}}
\end{fulllineitems}

\index{ccmsh command line option!-e, --emu lists all available emulators and gives emulator details by name}\index{-e, --emu lists all available emulators and gives emulator details by name!ccmsh command line option}

\begin{fulllineitems}
\phantomsection\label{01_how_to_use_it:cmdoption-ccmsh-e}\pysigline{\bfcode{-e}\code{}\code{,~}\bfcode{--emu}\code{~lists~all~available~emulators~and~gives~emulator~details~by~name}}
\end{fulllineitems}

\index{ccmsh command line option!-x, --xml provide path to XML file with emulation details}\index{-x, --xml provide path to XML file with emulation details!ccmsh command line option}

\begin{fulllineitems}
\phantomsection\label{01_how_to_use_it:cmdoption-ccmsh-x}\pysigline{\bfcode{-x}\code{}\code{,~}\bfcode{--xml}\code{~provide~path~to~XML~file~with~emulation~details}}
\end{fulllineitems}

\index{ccmsh command line option!-n, --now add to the "-x" argument to override emulation start date and execute test immediately}\index{-n, --now add to the "-x" argument to override emulation start date and execute test immediately!ccmsh command line option}

\begin{fulllineitems}
\phantomsection\label{01_how_to_use_it:cmdoption-ccmsh-n}\pysigline{\bfcode{-n}\code{}\code{,~}\bfcode{--now}\code{~add~to~the~''-x''~argument~to~override~emulation~start~date~and~execute~test~immediately}}
\end{fulllineitems}

\index{ccmsh command line option!-d, --delete delete emulation by name}\index{-d, --delete delete emulation by name!ccmsh command line option}

\begin{fulllineitems}
\phantomsection\label{01_how_to_use_it:cmdoption-ccmsh-d}\pysigline{\bfcode{-d}\code{}\code{,~}\bfcode{--delete}\code{~delete~emulation~by~name}}
\end{fulllineitems}

\index{ccmsh command line option!-p, --purge wipes all DB entries, removes all scheduled jobs and log files}\index{-p, --purge wipes all DB entries, removes all scheduled jobs and log files!ccmsh command line option}

\begin{fulllineitems}
\phantomsection\label{01_how_to_use_it:cmdoption-ccmsh-p}\pysigline{\bfcode{-p}\code{}\code{,~}\bfcode{--purge}\code{~wipes~all~DB~entries,~removes~all~scheduled~jobs~and~log~files}}
\end{fulllineitems}

\index{ccmsh command line option!--start launch Scheduler or API daemon}\index{--start launch Scheduler or API daemon!ccmsh command line option}

\begin{fulllineitems}
\phantomsection\label{01_how_to_use_it:cmdoption-ccmsh--start}\pysigline{\bfcode{--start}\code{~launch~Scheduler~or~API~daemon}}
\end{fulllineitems}

\index{ccmsh command line option!--stop stop Scheduler or API daemon}\index{--stop stop Scheduler or API daemon!ccmsh command line option}

\begin{fulllineitems}
\phantomsection\label{01_how_to_use_it:cmdoption-ccmsh--stop}\pysigline{\bfcode{--stop}\code{~stop~Scheduler~or~API~daemon}}
\end{fulllineitems}

\index{ccmsh command line option!--show show OS information on Scheduler or API daemon}\index{--show show OS information on Scheduler or API daemon!ccmsh command line option}

\begin{fulllineitems}
\phantomsection\label{01_how_to_use_it:cmdoption-ccmsh--show}\pysigline{\bfcode{--show}\code{~show~OS~information~on~Scheduler~or~API~daemon}}
\end{fulllineitems}



\subsection{REST API Description}
\label{01_how_to_use_it:rest-api-description}
\begin{Verbatim}[commandchars=\\\{\},numbers=left,firstnumber=1,stepnumber=1]
\PYG{n+nt}{\PYGZlt{}emulation}\PYG{n+nt}{\PYGZgt{}}
  \PYG{n+nt}{\PYGZlt{}name}\PYG{n+nt}{\PYGZgt{}}Emu\PYGZhy{}CPU\PYGZhy{}RAM\PYGZhy{}IO\PYG{n+nt}{\PYGZlt{}/name\PYGZgt{}}
  \PYG{n+nt}{\PYGZlt{}emulationType}\PYG{n+nt}{\PYGZgt{}}Mix\PYG{n+nt}{\PYGZlt{}/emulationType\PYGZgt{}}
  \PYG{n+nt}{\PYGZlt{}resourceType}\PYG{n+nt}{\PYGZgt{}}Mix\PYG{n+nt}{\PYGZlt{}/resourceType\PYGZgt{}}
  \PYG{n+nt}{\PYGZlt{}startTime}\PYG{n+nt}{\PYGZgt{}}now\PYG{n+nt}{\PYGZlt{}/startTime\PYGZgt{}}
  \PYG{c}{\PYGZlt{}!\PYGZhy{}\PYGZhy{}}\PYG{c}{duration in seconds }\PYG{c}{\PYGZhy{}\PYGZhy{}\PYGZgt{}}
  \PYG{n+nt}{\PYGZlt{}stopTime}\PYG{n+nt}{\PYGZgt{}}180\PYG{n+nt}{\PYGZlt{}/stopTime\PYGZgt{}}

  \PYG{n+nt}{\PYGZlt{}distributions}\PYG{n+nt}{\PYGZgt{}}
     \PYG{n+nt}{\PYGZlt{}name}\PYG{n+nt}{\PYGZgt{}}Distro1\PYG{n+nt}{\PYGZlt{}/name\PYGZgt{}}
     \PYG{n+nt}{\PYGZlt{}startTime}\PYG{n+nt}{\PYGZgt{}}5\PYG{n+nt}{\PYGZlt{}/startTime\PYGZgt{}}
     \PYG{c}{\PYGZlt{}!\PYGZhy{}\PYGZhy{}}\PYG{c}{duration in seconds }\PYG{c}{\PYGZhy{}\PYGZhy{}\PYGZgt{}}
     \PYG{n+nt}{\PYGZlt{}duration}\PYG{n+nt}{\PYGZgt{}}30\PYG{n+nt}{\PYGZlt{}/duration\PYGZgt{}}
     \PYG{n+nt}{\PYGZlt{}granularity}\PYG{n+nt}{\PYGZgt{}}3\PYG{n+nt}{\PYGZlt{}/granularity\PYGZgt{}}
     \PYG{n+nt}{\PYGZlt{}distribution} \PYG{n+na}{href=}\PYG{l+s}{\PYGZdq{}/distributions/linear\PYGZdq{}} \PYG{n+na}{name=}\PYG{l+s}{\PYGZdq{}linear\PYGZdq{}} \PYG{n+nt}{/\PYGZgt{}}
   \PYG{c}{\PYGZlt{}!\PYGZhy{}\PYGZhy{}}\PYG{c}{cpu utilization distribution range}\PYG{c}{\PYGZhy{}\PYGZhy{}\PYGZgt{}}
      \PYG{n+nt}{\PYGZlt{}startLoad}\PYG{n+nt}{\PYGZgt{}}90\PYG{n+nt}{\PYGZlt{}/startLoad\PYGZgt{}}
      \PYG{n+nt}{\PYGZlt{}stopLoad}\PYG{n+nt}{\PYGZgt{}}10\PYG{n+nt}{\PYGZlt{}/stopLoad\PYGZgt{}}
      \PYG{n+nt}{\PYGZlt{}emulator} \PYG{n+na}{href=}\PYG{l+s}{\PYGZdq{}/emulators/stressapptest\PYGZdq{}} \PYG{n+na}{name=}\PYG{l+s}{\PYGZdq{}lookbusy\PYGZdq{}} \PYG{n+nt}{/\PYGZgt{}}
      \PYG{n+nt}{\PYGZlt{}emulator\PYGZhy{}params}\PYG{n+nt}{\PYGZgt{}}
        \PYG{c}{\PYGZlt{}!\PYGZhy{}\PYGZhy{}}\PYG{c}{more parameters will be added }\PYG{c}{\PYGZhy{}\PYGZhy{}\PYGZgt{}}
        \PYG{n+nt}{\PYGZlt{}resourceType}\PYG{n+nt}{\PYGZgt{}}CPU\PYG{n+nt}{\PYGZlt{}/resourceType\PYGZgt{}}
   \PYG{c}{\PYGZlt{}!\PYGZhy{}\PYGZhy{}}\PYG{c}{Number of CPUs to keep busy (default: autodetected)}\PYG{c}{\PYGZhy{}\PYGZhy{}\PYGZgt{}}
   \PYG{n+nt}{\PYGZlt{}ncpus}\PYG{n+nt}{\PYGZgt{}}0\PYG{n+nt}{\PYGZlt{}/ncpus\PYGZgt{}}

      \PYG{n+nt}{\PYGZlt{}/emulator\PYGZhy{}params\PYGZgt{}}
  \PYG{n+nt}{\PYGZlt{}/distributions\PYGZgt{}}

   \PYG{n+nt}{\PYGZlt{}distributions}\PYG{n+nt}{\PYGZgt{}}
     \PYG{n+nt}{\PYGZlt{}name}\PYG{n+nt}{\PYGZgt{}}Distro2\PYG{n+nt}{\PYGZlt{}/name\PYGZgt{}}
     \PYG{n+nt}{\PYGZlt{}startTime}\PYG{n+nt}{\PYGZgt{}}5\PYG{n+nt}{\PYGZlt{}/startTime\PYGZgt{}}
     \PYG{c}{\PYGZlt{}!\PYGZhy{}\PYGZhy{}}\PYG{c}{duration in seconds }\PYG{c}{\PYGZhy{}\PYGZhy{}\PYGZgt{}}
     \PYG{n+nt}{\PYGZlt{}duration}\PYG{n+nt}{\PYGZgt{}}30\PYG{n+nt}{\PYGZlt{}/duration\PYGZgt{}}
     \PYG{n+nt}{\PYGZlt{}granularity}\PYG{n+nt}{\PYGZgt{}}3\PYG{n+nt}{\PYGZlt{}/granularity\PYGZgt{}}
     \PYG{n+nt}{\PYGZlt{}distribution} \PYG{n+na}{href=}\PYG{l+s}{\PYGZdq{}/distributions/linear\PYGZdq{}} \PYG{n+na}{name=}\PYG{l+s}{\PYGZdq{}linear\PYGZdq{}} \PYG{n+nt}{/\PYGZgt{}}
   \PYG{c}{\PYGZlt{}!\PYGZhy{}\PYGZhy{}}\PYG{c}{cpu utilization distribution range}\PYG{c}{\PYGZhy{}\PYGZhy{}\PYGZgt{}}
      \PYG{n+nt}{\PYGZlt{}startLoad}\PYG{n+nt}{\PYGZgt{}}10\PYG{n+nt}{\PYGZlt{}/startLoad\PYGZgt{}}
      \PYG{n+nt}{\PYGZlt{}stopLoad}\PYG{n+nt}{\PYGZgt{}}90\PYG{n+nt}{\PYGZlt{}/stopLoad\PYGZgt{}}
      \PYG{n+nt}{\PYGZlt{}emulator} \PYG{n+na}{href=}\PYG{l+s}{\PYGZdq{}/emulators/stressapptest\PYGZdq{}} \PYG{n+na}{name=}\PYG{l+s}{\PYGZdq{}lookbusy\PYGZdq{}} \PYG{n+nt}{/\PYGZgt{}}
      \PYG{n+nt}{\PYGZlt{}emulator\PYGZhy{}params}\PYG{n+nt}{\PYGZgt{}}
        \PYG{c}{\PYGZlt{}!\PYGZhy{}\PYGZhy{}}\PYG{c}{more parameters will be added }\PYG{c}{\PYGZhy{}\PYGZhy{}\PYGZgt{}}
        \PYG{n+nt}{\PYGZlt{}resourceType}\PYG{n+nt}{\PYGZgt{}}CPU\PYG{n+nt}{\PYGZlt{}/resourceType\PYGZgt{}}
   \PYG{c}{\PYGZlt{}!\PYGZhy{}\PYGZhy{}}\PYG{c}{Number of CPUs to keep busy (default: autodetected)}\PYG{c}{\PYGZhy{}\PYGZhy{}\PYGZgt{}}
   \PYG{n+nt}{\PYGZlt{}ncpus}\PYG{n+nt}{\PYGZgt{}}0\PYG{n+nt}{\PYGZlt{}/ncpus\PYGZgt{}}

      \PYG{n+nt}{\PYGZlt{}/emulator\PYGZhy{}params\PYGZgt{}}
  \PYG{n+nt}{\PYGZlt{}/distributions\PYGZgt{}}

  \PYG{n+nt}{\PYGZlt{}log}\PYG{n+nt}{\PYGZgt{}}
   \PYG{c}{\PYGZlt{}!\PYGZhy{}\PYGZhy{}}\PYG{c}{ Use value \PYGZdq{}1\PYGZdq{} to enable logging(by default logging is off)  }\PYG{c}{\PYGZhy{}\PYGZhy{}\PYGZgt{}}
   \PYG{n+nt}{\PYGZlt{}enable}\PYG{n+nt}{\PYGZgt{}}1\PYG{n+nt}{\PYGZlt{}/enable\PYGZgt{}}
   \PYG{c}{\PYGZlt{}!\PYGZhy{}\PYGZhy{}}\PYG{c}{ Use seconds for setting probe intervals(if logging is enabled default is 3sec)  }\PYG{c}{\PYGZhy{}\PYGZhy{}\PYGZgt{}}
   \PYG{n+nt}{\PYGZlt{}frequency}\PYG{n+nt}{\PYGZgt{}}3\PYG{n+nt}{\PYGZlt{}/frequency\PYGZgt{}}
  \PYG{n+nt}{\PYGZlt{}/log\PYGZgt{}}

\PYG{n+nt}{\PYGZlt{}/emulation\PYGZgt{}}
\end{Verbatim}


\section{Single Distribution: CLI Examples}
\label{02_cli_examples:single-distribution-cli-examples}\label{02_cli_examples::doc}
blah


\subsection{CPU}
\label{02_cli_examples:cpu}
blah


\subsection{I/O}
\label{02_cli_examples:i-o}
Words can have \emph{emphasis in italics} or be \textbf{bold} and you can
define code samples with back quotes, like when you talk about a
command: \code{sudo} gives you super user powers!


\subsection{Memory}
\label{02_cli_examples:memory}
blah


\subsection{Network}
\label{02_cli_examples:network}
blah

This is an example on how to link images:

\begin{Verbatim}[commandchars=\\\{\},numbers=left,firstnumber=1,stepnumber=1]
         \PYG{n+nt}{\PYGZlt{}emulation}\PYG{n+nt}{\PYGZgt{}}
           \PYG{n+nt}{\PYGZlt{}emuname}\PYG{n+nt}{\PYGZgt{}}CPU\PYGZus{}emu\PYG{n+nt}{\PYGZlt{}/emuname\PYGZgt{}}
           \PYG{n+nt}{\PYGZlt{}emuType}\PYG{n+nt}{\PYGZgt{}}Mix\PYG{n+nt}{\PYGZlt{}/emuType\PYGZgt{}}
           \PYG{n+nt}{\PYGZlt{}emuresourceType}\PYG{n+nt}{\PYGZgt{}}CPU\PYG{n+nt}{\PYGZlt{}/emuresourceType\PYGZgt{}}
           \PYG{n+nt}{\PYGZlt{}emustartTime}\PYG{n+nt}{\PYGZgt{}}now\PYG{n+nt}{\PYGZlt{}/emustartTime\PYGZgt{}}
           \PYG{c}{\PYGZlt{}!\PYGZhy{}\PYGZhy{}}\PYG{c}{duration in seconds }\PYG{c}{\PYGZhy{}\PYGZhy{}\PYGZgt{}}
           \PYG{n+nt}{\PYGZlt{}emustopTime}\PYG{n+nt}{\PYGZgt{}}15\PYG{n+nt}{\PYGZlt{}/emustopTime\PYGZgt{}}

           \PYG{n+nt}{\PYGZlt{}distributions}\PYG{n+nt}{\PYGZgt{}}
            \PYG{n+nt}{\PYGZlt{}name}\PYG{n+nt}{\PYGZgt{}}CPU\PYGZus{}distro\PYG{n+nt}{\PYGZlt{}/name\PYGZgt{}}
            \PYG{n+nt}{\PYGZlt{}startTime}\PYG{n+nt}{\PYGZgt{}}0\PYG{n+nt}{\PYGZlt{}/startTime\PYGZgt{}}
            \PYG{c}{\PYGZlt{}!\PYGZhy{}\PYGZhy{}}\PYG{c}{duration in seconds }\PYG{c}{\PYGZhy{}\PYGZhy{}\PYGZgt{}}
            \PYG{n+nt}{\PYGZlt{}duration}\PYG{n+nt}{\PYGZgt{}}10\PYG{n+nt}{\PYGZlt{}/duration\PYGZgt{}}
            \PYG{n+nt}{\PYGZlt{}granularity}\PYG{n+nt}{\PYGZgt{}}1\PYG{n+nt}{\PYGZlt{}/granularity\PYGZgt{}}
            \PYG{n+nt}{\PYGZlt{}distribution} \PYG{n+na}{href=}\PYG{l+s}{\PYGZdq{}/distributions/linear\PYGZdq{}} \PYG{n+na}{name=}\PYG{l+s}{\PYGZdq{}linear\PYGZdq{}} \PYG{n+nt}{/\PYGZgt{}}
            \PYG{c}{\PYGZlt{}!\PYGZhy{}\PYGZhy{}}\PYG{c}{cpu utilization distribution range}\PYG{c}{\PYGZhy{}\PYGZhy{}\PYGZgt{}}
            \PYG{n+nt}{\PYGZlt{}startLoad}\PYG{n+nt}{\PYGZgt{}}10\PYG{n+nt}{\PYGZlt{}/startLoad\PYGZgt{}}
            \PYG{n+nt}{\PYGZlt{}stopLoad}\PYG{n+nt}{\PYGZgt{}}95\PYG{n+nt}{\PYGZlt{}/stopLoad\PYGZgt{}}
            \PYG{n+nt}{\PYGZlt{}emulator} \PYG{n+na}{href=}\PYG{l+s}{\PYGZdq{}/emulators/lookbusy\PYGZdq{}} \PYG{n+na}{name=}\PYG{l+s}{\PYGZdq{}lookbusy\PYGZdq{}} \PYG{n+nt}{/\PYGZgt{}}
            \PYG{n+nt}{\PYGZlt{}emulator\PYGZhy{}params}\PYG{n+nt}{\PYGZgt{}}
                 \PYG{c}{\PYGZlt{}!\PYGZhy{}\PYGZhy{}}\PYG{c}{more parameters will be added }\PYG{c}{\PYGZhy{}\PYGZhy{}\PYGZgt{}}
                 \PYG{n+nt}{\PYGZlt{}resourceType}\PYG{n+nt}{\PYGZgt{}}CPU\PYG{n+nt}{\PYGZlt{}/resourceType\PYGZgt{}}
                 \PYG{c}{\PYGZlt{}!\PYGZhy{}\PYGZhy{}}\PYG{c}{Number of CPUs to keep busy (default: autodetected)}\PYG{c}{\PYGZhy{}\PYGZhy{}\PYGZgt{}}
                 \PYG{n+nt}{\PYGZlt{}ncpus}\PYG{n+nt}{\PYGZgt{}}0\PYG{n+nt}{\PYGZlt{}/ncpus\PYGZgt{}}
            \PYG{n+nt}{\PYGZlt{}/emulator\PYGZhy{}params\PYGZgt{}}
          \PYG{n+nt}{\PYGZlt{}/distributions\PYGZgt{}}

          \PYG{n+nt}{\PYGZlt{}log}\PYG{n+nt}{\PYGZgt{}}
            \PYG{c}{\PYGZlt{}!\PYGZhy{}\PYGZhy{}}\PYG{c}{ Use value \PYGZdq{}1\PYGZdq{} to enable logging(by default logging is off)  }\PYG{c}{\PYGZhy{}\PYGZhy{}\PYGZgt{}}
            \PYG{n+nt}{\PYGZlt{}enable}\PYG{n+nt}{\PYGZgt{}}1\PYG{n+nt}{\PYGZlt{}/enable\PYGZgt{}}
            \PYG{c}{\PYGZlt{}!\PYGZhy{}\PYGZhy{}}\PYG{c}{ Use seconds for setting probe intervals(if logging is enabled default is 3sec)  }\PYG{c}{\PYGZhy{}\PYGZhy{}\PYGZgt{}}
            \PYG{n+nt}{\PYGZlt{}frequency}\PYG{n+nt}{\PYGZgt{}}3\PYG{n+nt}{\PYGZlt{}/frequency\PYGZgt{}}
          \PYG{n+nt}{\PYGZlt{}/log\PYGZgt{}}
         \PYG{n+nt}{\PYGZlt{}/emulation\PYGZgt{}}
\end{Verbatim}


\section{Single Distribution: REST Examples}
\label{03_rest_examples::doc}\label{03_rest_examples:single-distribution-rest-examples}
blah


\subsection{CPU}
\label{03_rest_examples:cpu}
blah


\subsection{I/O}
\label{03_rest_examples:i-o}
Words can have \emph{emphasis in italics} or be \textbf{bold} and you can
define code samples with back quotes, like when you talk about a
command: \code{sudo} gives you super user powers!


\subsection{Memory}
\label{03_rest_examples:memory}
blah

blah


\subsection{Network}
\label{03_rest_examples:network}
blah

This is an example on how to link images:

\begin{Verbatim}[commandchars=\\\{\},numbers=left,firstnumber=1,stepnumber=1]
         \PYG{n+nt}{\PYGZlt{}emulation}\PYG{n+nt}{\PYGZgt{}}
           \PYG{n+nt}{\PYGZlt{}emuname}\PYG{n+nt}{\PYGZgt{}}CPU\PYGZus{}emu\PYG{n+nt}{\PYGZlt{}/emuname\PYGZgt{}}
           \PYG{n+nt}{\PYGZlt{}emuType}\PYG{n+nt}{\PYGZgt{}}Mix\PYG{n+nt}{\PYGZlt{}/emuType\PYGZgt{}}
           \PYG{n+nt}{\PYGZlt{}emuresourceType}\PYG{n+nt}{\PYGZgt{}}CPU\PYG{n+nt}{\PYGZlt{}/emuresourceType\PYGZgt{}}
           \PYG{n+nt}{\PYGZlt{}emustartTime}\PYG{n+nt}{\PYGZgt{}}now\PYG{n+nt}{\PYGZlt{}/emustartTime\PYGZgt{}}
           \PYG{c}{\PYGZlt{}!\PYGZhy{}\PYGZhy{}}\PYG{c}{duration in seconds }\PYG{c}{\PYGZhy{}\PYGZhy{}\PYGZgt{}}
           \PYG{n+nt}{\PYGZlt{}emustopTime}\PYG{n+nt}{\PYGZgt{}}15\PYG{n+nt}{\PYGZlt{}/emustopTime\PYGZgt{}}

           \PYG{n+nt}{\PYGZlt{}distributions}\PYG{n+nt}{\PYGZgt{}}
            \PYG{n+nt}{\PYGZlt{}name}\PYG{n+nt}{\PYGZgt{}}CPU\PYGZus{}distro\PYG{n+nt}{\PYGZlt{}/name\PYGZgt{}}
            \PYG{n+nt}{\PYGZlt{}startTime}\PYG{n+nt}{\PYGZgt{}}0\PYG{n+nt}{\PYGZlt{}/startTime\PYGZgt{}}
            \PYG{c}{\PYGZlt{}!\PYGZhy{}\PYGZhy{}}\PYG{c}{duration in seconds }\PYG{c}{\PYGZhy{}\PYGZhy{}\PYGZgt{}}
            \PYG{n+nt}{\PYGZlt{}duration}\PYG{n+nt}{\PYGZgt{}}10\PYG{n+nt}{\PYGZlt{}/duration\PYGZgt{}}
            \PYG{n+nt}{\PYGZlt{}granularity}\PYG{n+nt}{\PYGZgt{}}1\PYG{n+nt}{\PYGZlt{}/granularity\PYGZgt{}}
            \PYG{n+nt}{\PYGZlt{}distribution} \PYG{n+na}{href=}\PYG{l+s}{\PYGZdq{}/distributions/linear\PYGZdq{}} \PYG{n+na}{name=}\PYG{l+s}{\PYGZdq{}linear\PYGZdq{}} \PYG{n+nt}{/\PYGZgt{}}
            \PYG{c}{\PYGZlt{}!\PYGZhy{}\PYGZhy{}}\PYG{c}{cpu utilization distribution range}\PYG{c}{\PYGZhy{}\PYGZhy{}\PYGZgt{}}
            \PYG{n+nt}{\PYGZlt{}startLoad}\PYG{n+nt}{\PYGZgt{}}10\PYG{n+nt}{\PYGZlt{}/startLoad\PYGZgt{}}
            \PYG{n+nt}{\PYGZlt{}stopLoad}\PYG{n+nt}{\PYGZgt{}}95\PYG{n+nt}{\PYGZlt{}/stopLoad\PYGZgt{}}
            \PYG{n+nt}{\PYGZlt{}emulator} \PYG{n+na}{href=}\PYG{l+s}{\PYGZdq{}/emulators/lookbusy\PYGZdq{}} \PYG{n+na}{name=}\PYG{l+s}{\PYGZdq{}lookbusy\PYGZdq{}} \PYG{n+nt}{/\PYGZgt{}}
            \PYG{n+nt}{\PYGZlt{}emulator\PYGZhy{}params}\PYG{n+nt}{\PYGZgt{}}
                 \PYG{c}{\PYGZlt{}!\PYGZhy{}\PYGZhy{}}\PYG{c}{more parameters will be added }\PYG{c}{\PYGZhy{}\PYGZhy{}\PYGZgt{}}
                 \PYG{n+nt}{\PYGZlt{}resourceType}\PYG{n+nt}{\PYGZgt{}}CPU\PYG{n+nt}{\PYGZlt{}/resourceType\PYGZgt{}}
                 \PYG{c}{\PYGZlt{}!\PYGZhy{}\PYGZhy{}}\PYG{c}{Number of CPUs to keep busy (default: autodetected)}\PYG{c}{\PYGZhy{}\PYGZhy{}\PYGZgt{}}
                 \PYG{n+nt}{\PYGZlt{}ncpus}\PYG{n+nt}{\PYGZgt{}}0\PYG{n+nt}{\PYGZlt{}/ncpus\PYGZgt{}}
            \PYG{n+nt}{\PYGZlt{}/emulator\PYGZhy{}params\PYGZgt{}}
          \PYG{n+nt}{\PYGZlt{}/distributions\PYGZgt{}}

          \PYG{n+nt}{\PYGZlt{}log}\PYG{n+nt}{\PYGZgt{}}
            \PYG{c}{\PYGZlt{}!\PYGZhy{}\PYGZhy{}}\PYG{c}{ Use value \PYGZdq{}1\PYGZdq{} to enable logging(by default logging is off)  }\PYG{c}{\PYGZhy{}\PYGZhy{}\PYGZgt{}}
            \PYG{n+nt}{\PYGZlt{}enable}\PYG{n+nt}{\PYGZgt{}}1\PYG{n+nt}{\PYGZlt{}/enable\PYGZgt{}}
            \PYG{c}{\PYGZlt{}!\PYGZhy{}\PYGZhy{}}\PYG{c}{ Use seconds for setting probe intervals(if logging is enabled default is 3sec)  }\PYG{c}{\PYGZhy{}\PYGZhy{}\PYGZgt{}}
            \PYG{n+nt}{\PYGZlt{}frequency}\PYG{n+nt}{\PYGZgt{}}3\PYG{n+nt}{\PYGZlt{}/frequency\PYGZgt{}}
          \PYG{n+nt}{\PYGZlt{}/log\PYGZgt{}}
         \PYG{n+nt}{\PYGZlt{}/emulation\PYGZgt{}}
\end{Verbatim}


\section{Multiple Distribution: CLI Examples}
\label{04_cli_mult_examples:multiple-distribution-cli-examples}\label{04_cli_mult_examples::doc}
blah


\subsection{CPU}
\label{04_cli_mult_examples:cpu}
blah


\subsection{I/O}
\label{04_cli_mult_examples:i-o}
Words can have \emph{emphasis in italics} or be \textbf{bold} and you can
define code samples with back quotes, like when you talk about a
command: \code{sudo} gives you super user powers!


\subsection{Memory}
\label{04_cli_mult_examples:memory}
blah

blah


\subsection{Network}
\label{04_cli_mult_examples:network}
blah

This is an example on how to link images:

\begin{Verbatim}[commandchars=\\\{\},numbers=left,firstnumber=1,stepnumber=1]
         \PYG{n+nt}{\PYGZlt{}emulation}\PYG{n+nt}{\PYGZgt{}}
           \PYG{n+nt}{\PYGZlt{}emuname}\PYG{n+nt}{\PYGZgt{}}CPU\PYGZus{}emu\PYG{n+nt}{\PYGZlt{}/emuname\PYGZgt{}}
           \PYG{n+nt}{\PYGZlt{}emuType}\PYG{n+nt}{\PYGZgt{}}Mix\PYG{n+nt}{\PYGZlt{}/emuType\PYGZgt{}}
           \PYG{n+nt}{\PYGZlt{}emuresourceType}\PYG{n+nt}{\PYGZgt{}}CPU\PYG{n+nt}{\PYGZlt{}/emuresourceType\PYGZgt{}}
           \PYG{n+nt}{\PYGZlt{}emustartTime}\PYG{n+nt}{\PYGZgt{}}now\PYG{n+nt}{\PYGZlt{}/emustartTime\PYGZgt{}}
           \PYG{c}{\PYGZlt{}!\PYGZhy{}\PYGZhy{}}\PYG{c}{duration in seconds }\PYG{c}{\PYGZhy{}\PYGZhy{}\PYGZgt{}}
           \PYG{n+nt}{\PYGZlt{}emustopTime}\PYG{n+nt}{\PYGZgt{}}15\PYG{n+nt}{\PYGZlt{}/emustopTime\PYGZgt{}}

           \PYG{n+nt}{\PYGZlt{}distributions}\PYG{n+nt}{\PYGZgt{}}
            \PYG{n+nt}{\PYGZlt{}name}\PYG{n+nt}{\PYGZgt{}}CPU\PYGZus{}distro\PYG{n+nt}{\PYGZlt{}/name\PYGZgt{}}
            \PYG{n+nt}{\PYGZlt{}startTime}\PYG{n+nt}{\PYGZgt{}}0\PYG{n+nt}{\PYGZlt{}/startTime\PYGZgt{}}
            \PYG{c}{\PYGZlt{}!\PYGZhy{}\PYGZhy{}}\PYG{c}{duration in seconds }\PYG{c}{\PYGZhy{}\PYGZhy{}\PYGZgt{}}
            \PYG{n+nt}{\PYGZlt{}duration}\PYG{n+nt}{\PYGZgt{}}10\PYG{n+nt}{\PYGZlt{}/duration\PYGZgt{}}
            \PYG{n+nt}{\PYGZlt{}granularity}\PYG{n+nt}{\PYGZgt{}}1\PYG{n+nt}{\PYGZlt{}/granularity\PYGZgt{}}
            \PYG{n+nt}{\PYGZlt{}distribution} \PYG{n+na}{href=}\PYG{l+s}{\PYGZdq{}/distributions/linear\PYGZdq{}} \PYG{n+na}{name=}\PYG{l+s}{\PYGZdq{}linear\PYGZdq{}} \PYG{n+nt}{/\PYGZgt{}}
            \PYG{c}{\PYGZlt{}!\PYGZhy{}\PYGZhy{}}\PYG{c}{cpu utilization distribution range}\PYG{c}{\PYGZhy{}\PYGZhy{}\PYGZgt{}}
            \PYG{n+nt}{\PYGZlt{}startLoad}\PYG{n+nt}{\PYGZgt{}}10\PYG{n+nt}{\PYGZlt{}/startLoad\PYGZgt{}}
            \PYG{n+nt}{\PYGZlt{}stopLoad}\PYG{n+nt}{\PYGZgt{}}95\PYG{n+nt}{\PYGZlt{}/stopLoad\PYGZgt{}}
            \PYG{n+nt}{\PYGZlt{}emulator} \PYG{n+na}{href=}\PYG{l+s}{\PYGZdq{}/emulators/lookbusy\PYGZdq{}} \PYG{n+na}{name=}\PYG{l+s}{\PYGZdq{}lookbusy\PYGZdq{}} \PYG{n+nt}{/\PYGZgt{}}
            \PYG{n+nt}{\PYGZlt{}emulator\PYGZhy{}params}\PYG{n+nt}{\PYGZgt{}}
                 \PYG{c}{\PYGZlt{}!\PYGZhy{}\PYGZhy{}}\PYG{c}{more parameters will be added }\PYG{c}{\PYGZhy{}\PYGZhy{}\PYGZgt{}}
                 \PYG{n+nt}{\PYGZlt{}resourceType}\PYG{n+nt}{\PYGZgt{}}CPU\PYG{n+nt}{\PYGZlt{}/resourceType\PYGZgt{}}
                 \PYG{c}{\PYGZlt{}!\PYGZhy{}\PYGZhy{}}\PYG{c}{Number of CPUs to keep busy (default: autodetected)}\PYG{c}{\PYGZhy{}\PYGZhy{}\PYGZgt{}}
                 \PYG{n+nt}{\PYGZlt{}ncpus}\PYG{n+nt}{\PYGZgt{}}0\PYG{n+nt}{\PYGZlt{}/ncpus\PYGZgt{}}
            \PYG{n+nt}{\PYGZlt{}/emulator\PYGZhy{}params\PYGZgt{}}
          \PYG{n+nt}{\PYGZlt{}/distributions\PYGZgt{}}

          \PYG{n+nt}{\PYGZlt{}log}\PYG{n+nt}{\PYGZgt{}}
            \PYG{c}{\PYGZlt{}!\PYGZhy{}\PYGZhy{}}\PYG{c}{ Use value \PYGZdq{}1\PYGZdq{} to enable logging(by default logging is off)  }\PYG{c}{\PYGZhy{}\PYGZhy{}\PYGZgt{}}
            \PYG{n+nt}{\PYGZlt{}enable}\PYG{n+nt}{\PYGZgt{}}1\PYG{n+nt}{\PYGZlt{}/enable\PYGZgt{}}
            \PYG{c}{\PYGZlt{}!\PYGZhy{}\PYGZhy{}}\PYG{c}{ Use seconds for setting probe intervals(if logging is enabled default is 3sec)  }\PYG{c}{\PYGZhy{}\PYGZhy{}\PYGZgt{}}
            \PYG{n+nt}{\PYGZlt{}frequency}\PYG{n+nt}{\PYGZgt{}}3\PYG{n+nt}{\PYGZlt{}/frequency\PYGZgt{}}
          \PYG{n+nt}{\PYGZlt{}/log\PYGZgt{}}
         \PYG{n+nt}{\PYGZlt{}/emulation\PYGZgt{}}
\end{Verbatim}


\section{Multiple Distribution: REST Examples}
\label{05_rest_mult_examples::doc}\label{05_rest_mult_examples:multiple-distribution-rest-examples}
blah


\subsection{CPU}
\label{05_rest_mult_examples:cpu}
blah


\subsection{I/O}
\label{05_rest_mult_examples:i-o}
Words can have \emph{emphasis in italics} or be \textbf{bold} and you can
define code samples with back quotes, like when you talk about a
command: \code{sudo} gives you super user powers!


\subsection{Memory}
\label{05_rest_mult_examples:memory}
blah

blah


\subsection{Network}
\label{05_rest_mult_examples:network}
blah

This is an example on how to link images:

\begin{Verbatim}[commandchars=\\\{\},numbers=left,firstnumber=1,stepnumber=1]
         \PYG{n+nt}{\PYGZlt{}emulation}\PYG{n+nt}{\PYGZgt{}}
           \PYG{n+nt}{\PYGZlt{}emuname}\PYG{n+nt}{\PYGZgt{}}CPU\PYGZus{}emu\PYG{n+nt}{\PYGZlt{}/emuname\PYGZgt{}}
           \PYG{n+nt}{\PYGZlt{}emuType}\PYG{n+nt}{\PYGZgt{}}Mix\PYG{n+nt}{\PYGZlt{}/emuType\PYGZgt{}}
           \PYG{n+nt}{\PYGZlt{}emuresourceType}\PYG{n+nt}{\PYGZgt{}}CPU\PYG{n+nt}{\PYGZlt{}/emuresourceType\PYGZgt{}}
           \PYG{n+nt}{\PYGZlt{}emustartTime}\PYG{n+nt}{\PYGZgt{}}now\PYG{n+nt}{\PYGZlt{}/emustartTime\PYGZgt{}}
           \PYG{c}{\PYGZlt{}!\PYGZhy{}\PYGZhy{}}\PYG{c}{duration in seconds }\PYG{c}{\PYGZhy{}\PYGZhy{}\PYGZgt{}}
           \PYG{n+nt}{\PYGZlt{}emustopTime}\PYG{n+nt}{\PYGZgt{}}15\PYG{n+nt}{\PYGZlt{}/emustopTime\PYGZgt{}}

           \PYG{n+nt}{\PYGZlt{}distributions}\PYG{n+nt}{\PYGZgt{}}
            \PYG{n+nt}{\PYGZlt{}name}\PYG{n+nt}{\PYGZgt{}}CPU\PYGZus{}distro\PYG{n+nt}{\PYGZlt{}/name\PYGZgt{}}
            \PYG{n+nt}{\PYGZlt{}startTime}\PYG{n+nt}{\PYGZgt{}}0\PYG{n+nt}{\PYGZlt{}/startTime\PYGZgt{}}
            \PYG{c}{\PYGZlt{}!\PYGZhy{}\PYGZhy{}}\PYG{c}{duration in seconds }\PYG{c}{\PYGZhy{}\PYGZhy{}\PYGZgt{}}
            \PYG{n+nt}{\PYGZlt{}duration}\PYG{n+nt}{\PYGZgt{}}10\PYG{n+nt}{\PYGZlt{}/duration\PYGZgt{}}
            \PYG{n+nt}{\PYGZlt{}granularity}\PYG{n+nt}{\PYGZgt{}}1\PYG{n+nt}{\PYGZlt{}/granularity\PYGZgt{}}
            \PYG{n+nt}{\PYGZlt{}distribution} \PYG{n+na}{href=}\PYG{l+s}{\PYGZdq{}/distributions/linear\PYGZdq{}} \PYG{n+na}{name=}\PYG{l+s}{\PYGZdq{}linear\PYGZdq{}} \PYG{n+nt}{/\PYGZgt{}}
            \PYG{c}{\PYGZlt{}!\PYGZhy{}\PYGZhy{}}\PYG{c}{cpu utilization distribution range}\PYG{c}{\PYGZhy{}\PYGZhy{}\PYGZgt{}}
            \PYG{n+nt}{\PYGZlt{}startLoad}\PYG{n+nt}{\PYGZgt{}}10\PYG{n+nt}{\PYGZlt{}/startLoad\PYGZgt{}}
            \PYG{n+nt}{\PYGZlt{}stopLoad}\PYG{n+nt}{\PYGZgt{}}95\PYG{n+nt}{\PYGZlt{}/stopLoad\PYGZgt{}}
            \PYG{n+nt}{\PYGZlt{}emulator} \PYG{n+na}{href=}\PYG{l+s}{\PYGZdq{}/emulators/lookbusy\PYGZdq{}} \PYG{n+na}{name=}\PYG{l+s}{\PYGZdq{}lookbusy\PYGZdq{}} \PYG{n+nt}{/\PYGZgt{}}
            \PYG{n+nt}{\PYGZlt{}emulator\PYGZhy{}params}\PYG{n+nt}{\PYGZgt{}}
                 \PYG{c}{\PYGZlt{}!\PYGZhy{}\PYGZhy{}}\PYG{c}{more parameters will be added }\PYG{c}{\PYGZhy{}\PYGZhy{}\PYGZgt{}}
                 \PYG{n+nt}{\PYGZlt{}resourceType}\PYG{n+nt}{\PYGZgt{}}CPU\PYG{n+nt}{\PYGZlt{}/resourceType\PYGZgt{}}
                 \PYG{c}{\PYGZlt{}!\PYGZhy{}\PYGZhy{}}\PYG{c}{Number of CPUs to keep busy (default: autodetected)}\PYG{c}{\PYGZhy{}\PYGZhy{}\PYGZgt{}}
                 \PYG{n+nt}{\PYGZlt{}ncpus}\PYG{n+nt}{\PYGZgt{}}0\PYG{n+nt}{\PYGZlt{}/ncpus\PYGZgt{}}
            \PYG{n+nt}{\PYGZlt{}/emulator\PYGZhy{}params\PYGZgt{}}
          \PYG{n+nt}{\PYGZlt{}/distributions\PYGZgt{}}

          \PYG{n+nt}{\PYGZlt{}log}\PYG{n+nt}{\PYGZgt{}}
            \PYG{c}{\PYGZlt{}!\PYGZhy{}\PYGZhy{}}\PYG{c}{ Use value \PYGZdq{}1\PYGZdq{} to enable logging(by default logging is off)  }\PYG{c}{\PYGZhy{}\PYGZhy{}\PYGZgt{}}
            \PYG{n+nt}{\PYGZlt{}enable}\PYG{n+nt}{\PYGZgt{}}1\PYG{n+nt}{\PYGZlt{}/enable\PYGZgt{}}
            \PYG{c}{\PYGZlt{}!\PYGZhy{}\PYGZhy{}}\PYG{c}{ Use seconds for setting probe intervals(if logging is enabled default is 3sec)  }\PYG{c}{\PYGZhy{}\PYGZhy{}\PYGZgt{}}
            \PYG{n+nt}{\PYGZlt{}frequency}\PYG{n+nt}{\PYGZgt{}}3\PYG{n+nt}{\PYGZlt{}/frequency\PYGZgt{}}
          \PYG{n+nt}{\PYGZlt{}/log\PYGZgt{}}
         \PYG{n+nt}{\PYGZlt{}/emulation\PYGZgt{}}
\end{Verbatim}


\chapter{Indices and tables}
\label{index:indices-and-tables}\begin{itemize}
\item {} 
\emph{genindex}

\item {} 
\emph{search}

\end{itemize}



\renewcommand{\indexname}{Index}
\printindex
\end{document}
